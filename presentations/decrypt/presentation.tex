\documentclass{beamer}
\usetheme{CMU}

\usepackage{slpython}
\usepackage{underscore}

\usepackage{pgf,pgfarrows,pgfnodes,pgfautomata,pgfheaps,pgfshade}
\usepackage{amsmath,amssymb}
\usepackage[utf8]{inputenc}
\usepackage{colortbl}
\usepackage[english]{babel}
\usepackage{booktabs}

\newcommand*{\True}{\mbox{True}}
\newcommand*{\False}{\mbox{False}}
\newcommand*{\Expected}{\ensuremath{\mathbb{E}}}
\newcommand{\pd}[2]{\frac{\partial#1}{\partial#2}}
\newcommand{\pdd}[2]{\frac{\partial^2 #1}{\partial #2^2}} 

\newcommand*{\error}[1]{{\color{red}\textbf{#1}}}
\newcommand*{\Assign}{\ensuremath\,:=\,}
\newcommand*{\Dkl}{\ensuremath{D_{\text{\textsc{kl}}}}}
\newcommand*{\vect}[1]{{\vec{#1}}}

\title[Jug Tutorial 0]{Jug Tutorial: Coarse-Level Parallel Programming in Python}
\author{Luis Pedro Coelho}
\institute{lpc@cmu.edu}
%\date{}

\graphicspath{{figures/}{figures/generated/}{../images/}}

\AtBeginSection[] % Do nothing for \subsection*
{
	\begin{frame}<beamer>
		\frametitle{Outline}
		\tableofcontents[currentsection,currentsubsection]
	\end{frame}
}


\begin{document}
\frame{\titlepage}
\begin{frame}[fragile]
\frametitle{Problem}
Brute force a password.
\end{frame}


\begin{frame}[fragile]
\frametitle{Assumptions}
The \alert{crypt} module exists with elements:
\begin{itemize}
\item decrypt: decrtypt ciphertext given a password.
\item isgood: test whether this text could be the plaintext.
\item letters: just the letters.
\end{itemize}

Also, we know that the password is five letters.
\end{frame}

\begin{frame}[fragile]
\frametitle{Simple Solution}

\begin{python}
import itertools
from crypt import decode, letters, isgood, preprocess

ciphertext = file('secret.msg').read()
ciphertext = preprocess(ciphertext)

for p in itertools.product(letters, repeat=5):
    text = decode(ciphertext, p)
    if isgood(text):
        passwd = "".join(map(chr,p))
        print '%s:%s' % (passwd, text)
\end{python}
\end{frame}

\begin{frame}[fragile]

This solution does not take advantage of multiple processors.
\end{frame}


\begin{frame}[fragile]
\frametitle{Tasks}
\begin{python}
import itertools
from crypt import decode, letters, isgood, preprocess

ciphertext = file('secret.msg').read()
ciphertext = preprocess(ciphertext)

def decrypt(ciphertext, p):
    text = decode(ciphertext, p)
    if isgood(text):
        passwd = "".join(map(chr,p))
        return (passwd, text)
    # else: return None

results = []
for p in itertools.product(letters, repeat=5):
    results.append(Task(decrypt,ciphertext,p))
\end{python}
\end{frame}

\begin{frame}[fragile]
\frametitle{Python Magic}

\begin{python}
from jug import TaskGenerator
import itertools
from crypt import decode, letters, isgood, preprocess

ciphertext = file('secret.msg').read()
ciphertext = preprocess(ciphertext)

@TaskGenerator
def decrtypt(ciphertext, p):
    text = decode(ciphertext, p)
    if isgood(text):
        passwd = "".join(map(chr,p))
        return (passwd, text)
    # else: return None

results = []
for p in itertools.product(letters, repeat=5):
    results.append(decrypt(ciphertext,p))

\end{python}

\end{frame}

\begin{frame}[fragile]
\frametitle{Enter Jug}

You give it the Jugfile, it runs the tasks for you!

\end{frame}

\begin{frame}[fragile]
\frametitle{Jug Loop}

\begin{python}
while len(tasks) > 0:
    ready = [t for t in tasks if can_run(t)]
    for t in ready:
        if not is_running(t):
            t.run()
        tasks.remove(t)
\end{python}

Except jug is much fancier!

\end{frame}

\begin{frame}[fragile]
\frametitle{Jug Advantages}
\begin{enumerate}
\item Automatic task-level parallelization with dependency tracking.
\item Remember all intermediate results.
\item Makes writing parallel code look like writing sequential code.
\end{enumerate}
\end{frame}

\begin{frame}[fragile]
This example is actually not so good.

We have $26^5 \approx 11M$ tasks, all of which run very fast.

As a rule of thumb, your tasks should take at least a couple of seconds.
\end{frame}

\begin{frame}[fragile]

\alert{Solution} each task will be:

Given a letter, try \alert{all passwords} beginning with that letter.

Now, we have 26 tasks. Much better.
\end{frame}

\begin{frame}[fragile]
\begin{python}
@TaskGenerator
def decrypt(prefix):
    res = []
    for suffix in product(letters, repeat=5-len(prefix)):
        passwd = np.concatenate([prefix, suffix])
        text = decode(ciphertext, passwd)
        if isgood(text):
            passwd = "".join(map(chr, passwd))
            res.append( (passwd, text) )
    return res

@TaskGenerator
def join(partials):
    return list(chain(*partials))

results = join([decrypt([p]) for p in letters])
\end{python}
\end{frame}


\begin{frame}[fragile]

Let's call \alert{jug} now.

(Assuming our code is in a file called \alert{jugfile.py})
\end{frame}

\begin{frame}[fragile]
\begin{verbatim}
$jug status
Task name      Waiting  Ready  Finished Running
-----------------------------------------------
jugfile.join         1      0         0       0
jugfile.decrypt      0     26         0       0
...............................................
Total:               1     26         0       0

\end{verbatim}

Some tasks are ready. None are running.

\end{frame}
\begin{frame}[fragile]

\begin{verbatim}
$jug execute &
[1] 29501
$jug execute &
[2] 29502
\end{verbatim}

Executing in the background\dots

\end{frame}

\begin{frame}[fragile]
\begin{verbatim}
$jug status
Task name      Waiting  Ready  Finished Running
-----------------------------------------------
jugfile.join         1      0         0       0
jugfile.decrypt      0     24         0       2
...............................................
Total:               1     24         0       2

\end{verbatim}

Two tasks running. Good.
\end{frame}

\begin{frame}[fragile]
Wait a few minutes\dots
\end{frame}

\begin{frame}[fragile]
\begin{verbatim}
$jug status
Task name      Waiting  Ready  Finished Running
-----------------------------------------------
jugfile.join         1      0         0       0
jugfile.decrypt      0     14        10       2
...............................................
Total:               1     14        10       2
\end{verbatim}

Ten tasks have finished.

Notice how the \alert{join} task must wait for all the others.
\end{frame}

\begin{frame}[fragile]
A few more minutes\dots
\end{frame}

\begin{frame}[fragile]
\begin{verbatim}
$jug status
Task name      Waiting  Ready  Finished Running
-----------------------------------------------
jugfile.join         0      0         1       0
jugfile.decrypt      0      0        26       0
...............................................
Total:               0      0        27       0
\end{verbatim}
\end{frame}

\begin{frame}[fragile]
\frametitle{What Now?}

All tasks are finished!

How do I get to the results?
\end{frame} 

\begin{frame}[fragile]
\begin{python}
import jug
jug.init('jugfile', 'jugdata')
import jugfile
results = jug.task.value(jugfile.results)
for p,t in results:
        print "%s\n\n    Password was '%s'" % (t,p)
\end{python}

\alert{jug.init} is necessary to initialise the jug backend.

Then, the \alert{jugfile can be directly imported as a Python module}.

\end{frame}

\begin{frame}[fragile]
\frametitle{For more information}

\href{http://luispedro.org/software/jug}{http://luispedro.org/software/jug}

\href{http://github.com/luispedro/jug}{http://github.com/luispedro/jug}
\end{frame}
\end{document}

